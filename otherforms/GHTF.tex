%UCI core facility text from Melanie Oaks
\section{Campus Genomic Resources:} The UCI Genomics High-Throughput Facility (GHTF) has been in operation since 1999 and currently offers microarray and next generation DNA sequencing support for clients within and outside of the University of California system.  The staff has three years of experience in Illumina next generation sequencing.  Depending on client needs, the facility can prepare mRNA, small RNA, genomic and exome sequencing libraries. The staff has experience with different strategy options for exome enrichment, ribosomal RNA depletion and generation of multiplex libraries in order to insure maximal return of data in your experiments. The GHTF is equipped with Covaris S2 focused sonicating shearer; NanoDrop 1000 Spectrophotometer, MJ Research Tetrad thermalcycler, Agilent 2100 Bioanalyzer, and Agilent MX PRO RTPCR to facilitate the preparation of libraries and quality testing and final titration of samples.   Sequencing is currently performed using the Illumina HiSeq 2500 with a dual mode system for either high output or rapid mode sequencing. We also offer  Pac Bio RS II sequencing which provides very long read lengths that are useful for whole genome assembly, targeted sequencing and for studying base modifications. All sequences from GHTF are piped to the Institute for Genomics and Bioinformatics (IGB) where they are analyzed using Illumina pipeline or PacBio RS software under the supervision of Professor Pierre Baldi (UCI, Information and Computer Sciences; Director, IGB) .  Data are reported to users as aligned or not aligned fastq files depending on user specification.   Several software programs including Genomics Workbench (CLC Bio) and JMP Genomics (SAS) are installed in the GHTF for client data analysis.  Clients of the GHTF have access to the campus high performance computing cluster for analysis of sequences involving large amounts of data processing. The computer cluster is a shared computing cluster supervised by a GridEngine scheduler and has ~2000 64 bit cores including 25 nodes of 64 cores with an aggregate 8.8TB RAM and 500 TB storage in a Gluster distributed filesystem. Clients have priority on four 512 GB computing nodes with shared access to the rest of the cluster.   Under supervision of Professor Chad Garner, Ph.D., UCI Dept of Epidemiology, bioinformaticist, Jenny Wu, Ph.D., supports clients in post-pipeline analysis on a recharge basis.